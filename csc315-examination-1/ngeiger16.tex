
\chapter{Nicci Geiger}

\begin{enumerate}
  \item Most programming languages require the use of brackets to
    enclose the index in a reference to an element of an array.
  \begin{enumerate}
    \item Identify a language the requires the use of parentheses
      to enclose the index in a reference to an element of an array.
    \item Why did the designers of the language choose parentheses
      rather than brackets?
    \end{enumerate}

  \begin{answer}

  \begin{enumerate}
    \item Ada uses parentheses to enclose the index refrence to the
      element of an array.
    \item This was done for uniformity between array refrences and
      function call expressions because they are both mappings.
    \end{enumerate}

    \end{answer}
    
  \item What is the relationship between a lexeme and a token?

  \begin{answer}

    A lexemes is the lowest-level syntatic units.  Lexemes are
    patitioned into groups which are represented by tokens.
    \end{answer}

  \item
  \begin{enumerate}
    \item What kind of symbols are found at the internal nodes of a
      parse tree?
    \item What kind of symbols are found at the leaves of a parse tree?
    \end{enumerate}

  \begin{answer}

  \begin{enumerate}
    \item Nonterminal symobols are found at internal nodes.
    \item Terminal symbols  are found at leaves.
    \end{enumerate}

    \end{answer}


  \item One of the most significant contributions from the developers
    of ALGOL 60 also limited the success of that language. What was
    that contribution?

  \begin{answer}

  The contribution was BNF (Backus-Naur Form) a natural notation for
  describing syntax.

    \end{answer}

  \item What problem were the creators of Common LISP trying to solve?

  \begin{answer}

    They were trying solve the issue of lack of portability in
    programs written in various dialects.

    \end{answer}

  \item What is an ambiguous context free grammar?

  \begin{answer}

   A grammar that generates a sentential for that has two or more parse trees.

    \end{answer}

  \item Contrast the complexity of algorithms that can parse strings
    that conform to the most general kinds of context free grammars
    and the complexity of the algorithms that can parse strings that
    conform to the grammars of programming languages?

  \begin{answer}

    The complexity of context free grammars is $O(n^3)$, while the
    grammars of programing languages have a complexity of $O(n)$ due to
    the fact that context free grammar parsers must frequently be
    backed up and reparse part of what is being analyzed.

    \end{answer}

  \item Java represents characters with Unicode. It is the first
    widely used programming language with this feature. What is the
    significance of this feature?

  \begin{answer}

   It uses a set of charactes that covers characters for most of the
   worlds alphabet.  natural languages. This caters to the need for
   global computer communication.

    \end{answer}

  \item How does the binary coded decimal type differ from the
    floating point type?

  \begin{answer}

    Float points type represent real numbers but are only approximate
    and are stored in binary and represented in fractions or
    exponents, while decimal type stores a fixed number of decimal
    points and precicely store them with a restricted range.

    \end{answer}

  \item Identify a user-defined ordinal type in the Java programming
    language.

  \begin{answer}

    Java has enumeration types so long as it is after Java 1.4.

    \end{answer}

  \item Mathematicians and programmers might have different ideas
    about the precedence of Boolean operators. Explain.

  \begin{answer}

   Mathematicians have the OR and AND opperators with equal precedence
   while programmers have AND with a higher precedence than OR.

    \end{answer}

  \item Programmers should use \verb+===+ rather than \verb+==+ to
    test the equality of the values of two expressions in JavaScript. Why?

  \begin{answer}

    The \verb+===+ over \verb+==+ prevents coersion of one type into
    another such as a string that is numbers to be coerced into a
    number rather than being a string containing the number.

    \end{answer}

  \item Describe a hazard of allowing short-circuited evaluation
    of expressions and side effects in expressions at the same time.

  \begin{answer}

    Short-circuit evaluations used in an expression with part of the
    expression being a side effectthat is not evaluated. The side
    effect will only occur in complete evaluation of the whole
    ezxpression. This causes the side effects in expressions to allow
    for subtle errors.

    \end{answer}

  \item Briefly describe the three steps in the mark-sweep algorithm
    for garbage collection.

  \begin{answer}

   The first are cells in the heap have indicators set to garbage. The
   second is when every pointer in the program is traced to the heap
   and reachables are marked to not be garbage. The third is that all
   cells not marked as being used are returned to being useable space.

    \end{answer}

  \item What led Yukihiro Matsumoto to create the Ruby programming language?

  \begin{answer}

    There was a dissatisfaction with Perl and Python and the designers
    as both supported object oriented programing but neither were
    purely object oriented programing.

    \end{answer}

  \item What did Microsoft aim to achieve with its development of the
    C\# language?

  \begin{answer}

    C\# was aimed to be a component-based software development in the
    .NET framework.

    \end{answer}

  \end{enumerate}



\section{More questions for discussion and review.}

\begin{enumerate}
  \item The design of which machine influenced the design
    of the control statements in FORTRAN?

  \item How many different kinds of control statements
    must the designer of a programming language include
    in a language?

  \item What is the one question that applies in the
    design of all statements that allow selection or
    iteration?

  \item What is an advantage of requiring that
    the \textbf{then} and \textbf{else} clauses of
    an \textbf{if} statement be compound statements?

  \item How does the \textbf{switch} statement in C\#
    differ from the \textbf{switch} statement in Java?

  \item Distinguish between 2 statements in Ruby
    that correspond to Java's \textbf{switch} statement.

  \item Features of a programming language sometimes persist
    longer than a feature of computing hardware that inspired
    and supported that part of the language's design.
    Similarly, features of hardware sometimes persist longer
    than some parts of a language's design that were created
    to take advantage of that feature in hardware.

    Give examples.

  \item Who most famously warned of the dangers of using the
    \textbf{goto} statement? What did Donald Knuth have to
    say about the use of the \textbf{goto} statement?

  \item Describes Ada's \textbf{for} loop. Are there some
    kinds of iteration that might be easier in Ada than
    in Java? Easier in Java than in Ada?

  \item What does it mean to say that the guarded commands
    of Ada are non-deterministic?

  \item The header files in a C program contain function
    prototypes. What is a function prototype?

  \item Every method in a Ruby program belongs to a class.
    A programmer can place a definition of a method inside
    the definition of a class or outside of the definition
    of any class that the programmer writes. To which class
    does the method belong in the second case?

  \item Distinguish between positional and keyword parameters.

  \item Ruby blocks are closures. What does that mean?

  \item What is a pure function?

  \item Some languages give programmers means to define
    both functions and procedures. Java doe not. Is that
    a serious limitation?

  \item Declarations of formal parameters in an Ada procedure
    can include, in addition to the names and types of the
    parameters, reserved words that do not appear in declarations
    in Java programs. 
    What is the purpose of those reserved words?
 
  \item The C language imposes a constraint upon programmers
    who want to pass a multidimensional array to a function.
    What is the constraint? How did the design of the Java
    programming language eliminate that constraint for 
    programmers who use that language?

  \item An activation record contains a return
    address, a dynamic link, parameters, and
    local variables.
  \begin{enumerate}
    \item To what does the return address point?
    \item To what does the dynamic link point?
    \end{enumerate}

  \item The stack will contain multiple activation
    records for a single subprogram under what
    circumstances?

  \item How (or why?) does the LIFO protocol apply to
    calls to and returns from subprograms?

  \item Which important development in computer architecture
    has changed the way that the stack is used in some
    systems for facilitating calls to and returns from
    subprograms?

  \item A dynamic chain contains a history of what?

  \item Which two numbers are needed to compute
    the address of a local variable in a subprogram?

  \item How does a Ruby module differ from a class?

  \item Memory for variables can be allocated on the heap
    and on the stack. In which place or places is memory
    allocated for objects in C++? in Java?

  \item What problems were solved by the addition
    of genericity to Java?

  \item What is the purpose of the static chain?

  \item What is a singleton?

  \item What are the two parts of the definition 
    of an abstract data type?

  \end{enumerate}


